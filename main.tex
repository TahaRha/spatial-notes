\documentclass[10pt,a4paper]{article}
\usepackage[margin=0.5in]{geometry}
\usepackage{amsmath, amssymb}
\usepackage{graphicx} % Add images
\usepackage{multicol} % Add two-column layout
\usepackage{enumitem}
\setlist{nolistsep}

\title{Notes on SpaceOS \& Path-trailing}
\author{Taha Rhaouti}
\date{April, 2024}

\begin{document}
\maketitle

\begin{multicols}{2}

    \section*{Objectives}
    These notes are intended to tackle the safety, design, and implementation aspects of SpaceOS, a spatially-aware operating system. We will explore its objectives, research areas, potential applications, and inspiration from the Spatial Web.

    \section*{Overview}
    SpaceOS is a proof of concept for a new operating system that leverages spatial configurations to optimize computing performance. This project explores the application of transportation theory, facility location problems, and other advanced mathematical techniques the path-trailing part of the system. The goal is to learn/implement caching, and resource management in distributed systems and networks.

    \section*{Objectives}
    \begin{itemize}
        \item Investigate optimizations in compute speed and caching by utilizing the spatial attributes of the operating system.
        \item Explore solutions ranging from simple discrete problems solvable using Linear Programming to more complex problems involving differential geometry and probabilistic methods.
    \end{itemize}

    \section*{Research Areas}
    \begin{itemize}
        \item \textbf{Transportation Theory}: Applying principles of transportation theory to optimize data and resource movement within the system.
        \item \textbf{Facility Location Problems}: Investigating how facility location algorithms can improve system efficiency and reduce redundancy.
        \item \textbf{Differential Geometry}: Utilizing concepts from differential geometry to handle complex spatial configurations and resource-tolling calculations.
        \item \textbf{Probabilistic Methods}: Implementing probabilistic "games" to simplify and enhance resource management and computational efficiency.
    \end{itemize}

    \section*{Potential Applications}
    \begin{itemize}
        \item \textbf{Compute Speed Optimization}: Reducing computation times through efficient spatial data management.
        \item \textbf{Caching and Redundancy Minimization}: Enhancing data caching mechanisms to minimize redundancy and improve access times.
        \item \textbf{Resource Management}: Optimizing resource allocation and usage in distributed systems and networks.
    \end{itemize}

    \section*{Inspiration from Spatial Web}
    A stateful Spatial Web enables smart digital twins of people, physical spaces, and objects to be reliably and securely linked together, spatially. The effect of this is that when an object or person moves into or out of any physical or virtual space, a Spatial Contract can be executed automatically, subject to a set of spatial permissions set by the owner or approved entity triggering a record of the action and/or initiating a transaction. This makes the Spatial Web a trustworthy network for any form of interaction, transaction, or transportation. 

    \subsection*{Smart Spaces}
    A Smart Space is a defined location— a virtual or physical “place” described by its boundaries, some descriptive and classification information, a Spatial Domain, and a set of interaction and transactional rules (Smart Contracts). Smart Spaces are “programmable space.” They are semantically aware and can reference and validate the permissions related to users or assets within them. They can be securely encrypted by Distributed Ledgers and can control what users, objects, software, or robotics are able to be used.

    \subsection*{Problem and Solution}
    \textbf{Problem}: Currently, there is no way to reliably assign Spatial Rights or Permissions management for Users, AI, Spatial Content, or IoT devices because there is no standard method to identify, locate, and assign permissions for activities.\\
    \textbf{Solution}: Enable any space to become a Smart Space whose boundaries are defined by coordinates—either real-world (latitude, longitude, and elevation/altitude) with 0,0,0 or virtual (x/y/z) including outdoor and indoor spaces. Enable for sub-millimeter granularity and third-party re-localization optimization. Smart Spaces enable assets to have proof of their location, ownership, and permissions in time and space, across any device, platform, and location within virtual spaces and in the real world. They are searchable and can transact with Users or Assets. They can support multiple Users and Channels of Spatial Content.\\
    \textbf{Benefit}: This solution enables multiple users to search, track, interact, and collaborate with Smart Assets across time and space within Smart Spaces (i.e., in virtual and geo-locations). Smart Spaces are programmable.

    \subsection*{Example}
    A couple interested in buying a home in another state virtually walks through the various rooms of potential houses and can place their furniture in it to see how it fits. The Port of Long Beach notifies a buyer’s account that the cargo that left Hong Kong has just arrived. The buyer’s account automatically pays the shipper minus the port fees.

    \section*{SpaceOS Structure}
    \textbf{Space Creation}\\
    \texttt{x = create space attributes}\\
    \texttt{y = create space attributes}\\
    \texttt{attributes = \{ loc: [x, y], size: [w, h], type: \_, …\}}

    \textbf{Path Creation}\\
    \texttt{p = create path attribs\_p}\\
    \texttt{attribs\_p =  \{start: x, end: y\}}

    \textbf{Path-Trail Mapping}\\
    To construct the path, we introduce WAYPOINTS, which can be introduced in two different ways: (a) an array of space instances, (b) specifying a space factory that produces the space instances.\\
    Space locations have definite locations and sizes. Space factories can be parameterized to construct space instances that have specific characteristics.

    \textbf{Example}\\
    \texttt{p = create path attribs\_p = \{start: x, end: y, waypoints: [a, b, c], ..\}}\\
    \texttt{a = create space attribs}\\
    \texttt{b=…, c=…}\\
    Factory z is generating the waypoints automatically.\\
    \texttt{q = create path attribs\_p = \{start: x, end: y, waypoints: z, …\}}

    \textbf{Constraints for Space Factory}\\
    \texttt{z = create spacefactory attribs\_f}\\
    \texttt{attribs\_f = \{xline: [N, M], yline: [N, M], type: \_, xstart: \_, xstep: \_, ystart: \_, ystep: \_, …\}}\\
    The space factory constrains the waypoints. It can also create multiple paths, connecting the spaces for load balance, minimum hazard paths, etc.

    \textbf{Space Classification}\\
    A space is directly created using a space factory. We can specify a main function to execute in a space. This is the controller. Each space runs a single controller while it can have many services. Spaces can be imaginary (i-space) or real (r-space).

    \textbf{Imaginary Spaces (i-space)}\\
    These are spaces that do not have a concrete physical realization. A portion of a physical space can be set as an i-space for a certain task.

    \textbf{Real Spaces (r-space)}\\
    These associate with the real physical world and can be closed or open (permission management, encryption). For example, a parking space or storage locker.

\end{multicols} % End two-column layout

\end{document}
